\chapter*{Resumen}
\addcontentsline{toc}{chapter}{Resumen}

Los tumores pulmonares son el tipo de cáncer más común, contabilizándose en 2018 2.1 millones de casos nuevos y 1.8 millones de muertes. Es además uno de los más letales al fallecer más de la mitad de los pacientes en el término de un año desde que son diagnosticados.

El cáncer de pulmón no microcítico (CPNM) es el tipo más frecuente, representando alrededor del 80–90\% de todos los casos. Más del 50\% de estos pacientes son diagnosticados en una etapa avanzada o metastásica, imposibilitando un tratamiento quirúrgico eficaz. En este contexto, la terapia dirigida se ha postulado como el tratamiento de primera línea.

Actualmente, el manejo óptimo del CPNM requiere una examinación periódica de los tumores con el fin de detectar biomarcadores predictivos y pronósticos específicos. Sin embargo, las biopsias de tejido son un procedimiento invasivo limitado a ciertas localizaciones, no bien acogidas en la práctica clínica y generalmente incapaces de reflejar la heterogeneidad del tumor. Ante estas limitaciones han emergido las técnicas de biopsia líquida, que emplean diferentes fluidos corporales (sangre periférica, líquido cefalorraquídeo, etc.) de pacientes con cáncer como fuente del material derivado de los tumores.

A lo largo de la última década, la identificación de subconjuntos moleculares con sensibilidades únicas a fármacos específicos ha transformado completamente el tratamiento de los pacientes con adenocarcinoma pulmonar ALK+, favoreciendo el avance hacia una medicina más personalizada. La terapia molecular es ahora el estándar en estos pacientes, mejorándose su calidad de vida y supervivencia. No obstante, el beneficio a largo plazo de estas terapias está limitado por la aparición de resistencias adquiridas a los fármacos administrados. Por ello, se hace evidente la necesidad de tecnologías de seguimiento en tiempo real que guíen el tratamiento a nivel molecular tras el surgimiento de mutaciones de resistencia.

En este contexto, este estudio ha conseguido incorporar a la práctica clínica una forma automática de filtrar y seleccionar variantes genéticas mediante el análisis e interpretación de los datos obtenidos de la secuenciación de nueva generación (NGS). El algoritmo desarrollado ha sido capaz de identificar desviaciones genéticas clínicamente relevantes en muestras de biopsia líquida analizadas por NGS, permitiendo a los oncólogos actuar más decididamente sobre la información genómica de cada caso. Actualmente, el análisis de los datos brutos obtenidos del secuenciador no está estandarizado y antes de la implementación del algoritmo se realizaba manualmente, con todos los inconvenientes que conllevaba.

El proceso de diseño y evaluación del algoritmo se realizó en una población de estudio de 30 pacientes con CPNM avanzado. Después de realizarse una confirmación digital por PCR, que identificó a 8 sujetos con mutaciones en el gen ALK, el algoritmo mostró una precisión del 82.95\% (para una prevalencia de mutaciones ALK del 20\%), siendo capaz de detectar las alteraciones genéticas en 7 (87.5 \%) de los 8 pacientes ALK+.

En definitiva, este estudio ha logrado combinar eficazmente los procedimientos clínicos y el análisis bioinformático, teniendo un impacto directo en múltiples niveles: (i) para detectar mutaciones conductoras; (ii) para identificar mutaciones de resistencia; (iii) para monitorizar la respuesta a las terapias; y (iv) para realizar un pronóstico.

\subsection*{Palabras clave}

Biopsia líquida, ADN tumoral circulante (ctDNA), cáncer de pulmón no microcítico (CPNM), terapia dirigida, cinasa del linfoma anaplásico (ALK), secuenciación de nueva generación (NGS), PCR digital (dPCR), bioinformática