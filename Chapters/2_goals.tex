\chapter{Research Goals}

Diagnosis and targeting of ALK-positive NSCLC now form part of clinical routine, and the results obtained with ALK inhibitors are unprecedented considering the PFS and OS. However, although the development of novel drugs will continue to be a priority of cancer treatment, there is a need for better prognostic tools to assess therapeutic response, monitor tumor progression, and predict clinical outcomes.

In this context, the main objective of this thesis is to develop a new tool to better stratify patients, improving the current analysis of sequenced data from liquid biopsy samples. Eventually, this will allow oncologists to have more reliable information on the patient's tumor phenotype. Therefore, it will be possible to offer more individualized and optimal treatment to patients, without the risks associated with tissue biopsies.

This goal has led to the use in this study of non-invasive methods of describing tumor phenotype, including ctDNA and exosomes, to filter driver mutations in the ALK gene. Specifically, with the combination of molecular techniques and bioinformatics, this master's thesis aims to:
\begin{outline}
    \1 Establish analytical conditions for the study of mutations in the ALK gene from circulating DNA and exosomes in NSCLC advanced patients.
    \1 Perform a clinical study of mutations from circulating DNA and exosomes in a cohort of patients with advanced NSCLC.
    \1 Develop an automatic algorithm for filtering somatic mutations from sequenced data from ALK-positive NSCLC samples confirmed by PCR.
    \1 Estimate the utility of NGS and the developed algorithm for:
        \2 Detecting oncogenic drivers.
        \2 Identifying resistance mutations.
        \2 Monitoring therapy response.
        \2 Predicting clinical outcomes.
    \1 Assess the utility of liquid biopsy for the study of somatic mutations in the ALK gene from free circulating DNA and exosomes.
\end{outline}