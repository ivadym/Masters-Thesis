\chapter{Research Goals}

Diagnosis and targeting of ALK-positive NSCLC now form part of clinical routine, and the results obtained with ALK inhibitors are unprecedented considering the PFS and OS. However, although the development of novel drugs will continue to be a priority of cancer treatment, there is a need for better prognostic tools to assess therapeutic response, monitor tumor progression, and predict clinical outcomes.

In this context, the main objective of this thesis is to develop a new tool to better stratify patients, improving the current analysis of sequenced data from liquid biopsy samples. Eventually, this will allow oncologists to have more reliable information on the patient's tumor phenotype, favoring the introduction of more individualized and optimal treatments in clinical practice, without the risks associated with tissue biopsies.

This goal has led to the use in this study of non-invasive methods of describing tumor phenotype to filter driver mutations in the ALK gene, including ctDNA and exosomes. Specifically, with the combination of molecular techniques and bioinformatics, this master's thesis aims to:
\begin{outline}
    \1 Study the patterns of previously sequenced ALK-positive NSCLC samples confirmed by dPCR from the biobank of the Hospital Universitario Puerta de Hierro.
    \1 Develop an automatic algorithm for filtering somatic mutations in the ALK gene.
    \1 Perform a clinical study of mutations from circulating DNA and exosomes in a cohort of patients with ALK-positive advanced NSCLC confirmed by an anatomic pathology report (FISH and\slash or IHC).
    \1 Estimate the utility of NGS and the developed algorithm for:
        \2 Detecting oncogenic drivers.
        \2 Identifying resistance mutations.
        \2 Monitoring therapy response.
        \2 Predicting clinical outcomes.
    \1 Assess the utility of liquid biopsy for the study of somatic mutations in the ALK gene from free circulating DNA and exosomes.
\end{outline}