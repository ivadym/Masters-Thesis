\chapter{Materials and Methods}

\section{Patients}

\section{Tumor Samples}

\section{Mutational Analysis: ALK}


% 9_ Liquid biopsy in resistance
% In a study, a tracing blood test was performed on two advanced NSCLC patients with crizotinib treatment, and found that the ctDNA in serum and the allele frequency of ALK rear- rangements in ctDNA significantly increased after 10 months of medication, suggesting that the treatment failed.79 Three resistance mutations, L1152R, I1171T and L1196M, could be detected in the ctDNA of one of the patients, indicating that the change of nucleic acid sequence and blood concentration of ctDNA can be sensitive to detect the resistance. At this point, the second generation of ALKi should be put into use. The experimental data showed that the ORR of the second- generation ALKi, ceritinib and alectinib, after the treatment with crizotinib failed, were 56% and 50% respectively.22,80 Especially, alectinib had a significant effect on secondary resistance caused by L1196R.36 The other patient with central nervous system metastasis who had not used crizotinib in the same study sequentially used the second-generation ALKi ceritinib and alectinib, and had progress of metastases 1 year later.81 The ALK G1202R mutation was detected in ctDNAs. The clinical manifestations and the results of ctDNA detection pointed out the occurrence of resistance, and then the patient switched to the third-generation ALKi lorlatinib. The clinical symptoms caused by metastases improved, and the concentra- tion of ctDNA of G1202R in peripheral blood also decreased, showing the significance of ctDNA in resistance surveillance and guiding follow-up therapy. It is particularly important to detect ctDNA mutation after using the second-generation ALKi resistance. The second-generation ALKi can be used for patients who had or did not have the resistant mutation of the first-generation ALKi. While the presence of new secondary resistance mutation in ALK fusion gene, such as G1202R, is an essential indication of lorlatinib, the third-generation ALKi, otherwise, lorlatinib will not work.82