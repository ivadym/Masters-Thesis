\chapter{Materials and Methods}

\section{Patient characterization}

This study is part of the research project entitled "Clinical utility of liquid biopsy in non-small cell lung cancer patients with EML4-ALK translocation". Its objective is to determine the best strategy for the identification of EML4-ALK translocations and to assess the clinical utility of periodic tumor monitoring in ALK-positive NSCLC patients from liquid biopsies. In this context, between June 2015 and July 2019, several ALK-positive NSCLC advanced subjects were recruited from 6 hospitals across Spain, including the Hospital Universitario Puerta de Hierro, the Complejo Hospitalario Universitario A Coruña, the Hospital Universitario Fundación Jiménez Díaz, the Hospital Clínic Barcelona, the Hospital General Universitario de Alicante, and the Hospital General Universitario de Valencia.

The primary study was conducted under the precepts of the Helsinki Declaration and was approved by the Hospital Puerta de Hierro Ethics Committee (internal code 79-18). Patients were eligible if they consented to allow their clinical information to be used in the previously mentioned research project. Their clinical history was queried for information on the age of diagnosis, sex, smoking status, histology, Eastern Cooperative Oncology Group (ECOG) status at the start of the study, stage, previous therapies, and date of death.

Eligible patients had histologically confirmed diagnosis of stage III-IV NSCLC that was ALK-positive, a measurable disease according to the response evaluation criteria in solid tumors (RECIST, version 1.1), were candidates to be treated with an ALK inhibitor, and were 18 years old or older. Exclusion criteria included the impossibility of frequent venipuncture or evidence of any other major clinical disorder or finding that would have made it undesirable for the patient to participate in the study.

\section{Laboratory Procedures}



\subsection{Sample collection}
% CCLM
% In all cases, whole blood samples were collected in an 8.5 mL PPTTM tube (Becton Dickinson) containing a gel barrier to separate the plasma after centrifugation.