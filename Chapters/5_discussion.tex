\chapter{Discussion and Conclusions}

% The role of TP53 co-mutations in ALK-rearranged NSCLC is usually associated with increased degrees of genomic copy number instability and a higher somatic mutation burden \cite{NSCLC_alterations}. Therefore, co-occurring TP53 mutations predict an unfavorable outcome of therapies with tyrosine kinase inhibitors, and in general of systemic therapies.

% These resistances to second-generation ALK TKIs revealed the need for a change in treatment to ALK inhibitors such as brigatinib and lorlatinib, which target G1202R mutations (\autoref{tab:ALK_inhibitors})

% and may be sensitive to treatments with ceritinib, alectinib, and lorlatinib \cite{ALK_resistance, ALK_inhibitors}.

% In the latter case, any future ALK or EGFR mutations were also discarded since KRAS alterations seem to exclude them \cite{Mol_bio, NSCLC_therapies}.


% \section{Utility}

% A key benefit for physicians is the serial use of digital PCR to monitor one or more driver mutations throughout treatment to determine response and recurrence since the level of mutated DNA found in liquid biopsies has been found to reflect the size of the tumor(s). It is also possible to use NGS to track progress; however, in these cases it is less practical due to cost and longer turnaround time.

%Conclusions: The use of dPCR allows the detection of low-frequency tumor-specific mutations in plasma with a greater sensitivity compared to NGS providing the high accuracy and sensitivity essential for liquid biopsies.





% \section{Discussion}

% You should include the following information:
% 
% The major findings of your study
% The meaning of those findings
% How these findings relate to what others have done
% Limitations of your findings
% An explanation for any surprising, unexpected, or inconclusive results
% Suggestions for further research
% Your discussion should NOT include any of the following information:
% 
% New results or data not presented previously in the paper
% Unwarranted speculation
% Tangential issues
% Conclusions not supported by your data


% \section{Conclusions}


% Finally, it should be noted that to test the real effectiveness of the algorithm it would be necessary to analyze a more extensive and external data set. This is because the developed pipeline may correspond too closely or exactly to the particular mutations in this study, and may therefore fail to fit additional data or predict future observations reliably.


% Your conclusion should:
% 
% Restate your hypothesis or research question
% Restate your major findings
% Tell the reader what contribution your study has made to the existing literature
% Highlight any limitations of your study
% State future directions for research/recommendations
% Your conclusion should NOT:
% 
% Introduce new arguments
% Introduce new data
% Fail to include your research question
% Fail to state your major results


%\section{Future Directions}
