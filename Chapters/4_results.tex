\chapter{Results}

\section{Cohort of patients}

% TLCR
% Patients were followed from their diagnosis of stage IV disease.

% CCLM
% A total of 54 samples were obtained from 52 NSCLC patients and two healthy donors, after signing the appropriate informed consent.

% TESIS estela
% Para establecer la mejor estrategia metodológica, para la identificación del paciente candidato a recibir un inhibidor de ALK, se emplearán 20 muestras pre-tratamiento, pareadas de plasma y plaquetas, de pacientes con cáncer de pulmón no microcítico con enfermedad avanzada y translocación de ALK documentada según los protocolos asistenciales. Estas muestras se encuentran disponibles en el Biobanco del Hospital Universitario Puerta de Hierro.
% A fin de evaluar la utilidad de la biopsia líquida para monitorizar al paciente con translocación de ALK se reclutarán 30 pacientes, a razón de 15 pacientes por año. De cada paciente se recogerán muestras a los 2, 4, 6, 8, 12, 15 y 18 meses de tratamiento y a la progresión.
% % Los datos demográficos, clínico-patológicos, el estado mutacional del tumor, así como el estado funcional de los pacientes fueron obtenidos de los informes médicos. El ajuste de dosis y el cambio de medicación fueron documentados a lo largo del estudio.
%VARIABLES
% -Progresión de la enfermedad según criterios RECIST.
% -Supervivencia libre de progresión desde el inicio del tratamiento.
% -Supervivencia global desde el inicio del tratamiento.
% -Tasa de respuesta obtenida
% -Status EML4-ALK (positivo o negativo) en la biopsia líquida.
% -Niveles plasmáticos de EML4-ALK

% ALK paper
% In total, 33 plasma and 2 cerebrospinal fluid (CSF) specimens were collected and analyzed. Samples were collected at the time of disease progression which was assessed according to RECIST criteria v.1. 






\section{}

% Análisis estadístico
%Para el estudio estadístico se contará con la ayuda de la unidad de bio-estadística del Instituto de Investigación Sanitaria Puerta de Hierro.
% Los resultados de frecuencia se expresarán en términos absolutos, en porcentajes e intervalos de confianza y, las variables cuantitativas como media ± desviación estándar o mediana (rango) según proceda.
% Para evaluar la concordancia del status de la translocación (positivo/negativo) entre las muestras de sangre (plaquetas y plasma) pre-tratamiento y el tejido se empleará el coeficiente Kappa de Cohen y su intervalo de confianza. Además, se analizará la sensibilidad, especificidad, valor predictivo positivo y valor predictivo negativo de la detección de la translocación en plasma (exosomas) y plaquetas, tomando como gold standar el resultado del tejido.
% Para analizar los datos de supervivencia de los pacientes se tomará como fecha de inicio, la fecha de inicio al tratamiento en primera línea y se recabarán los datos de: fecha de progresión, de muerte o fecha de última visita al oncólogo. La supervivencia libre de progresión se calculará desde la fecha de inicio al tratamiento en primera línea. Para el análisis de PFS y SG se usará el método de Kaplan-Meier y como estadístico de contraste se usará el Log- Rank. Los valores de los HR obtenidos serán ajustados por las variables clínico-patologícas pertinentes.
%Las pruebas con un valor de p <0.05 serán consideradas estadísticamente significativas.